\documentclass[dvipdfmx]{jsarticle}
\usepackage[dvipdfmx]{graphicx}
\graphicspath{{ピクチャ/}}
\usepackage{otf}
\usepackage[dvipdfmx]{graphicx}
\usepackage{amsmath,amssymb}
\usepackage{url}
\newcommand{\e}{{\rm e}}
\newcommand{\dash}{^\prime}
\newcommand{\ddash}{^{\prime\prime}}
\newcommand{\f}{\frac}
\newcommand{\diff}{\mathrm{d}}
\newcommand{\dis}{\displaystyle}
\usepackage{tcolorbox}
\tcbuselibrary{skins}
\begin{document}

\title{\huge 岩大院試 重要事項まとめ}
\maketitle

\part{微分方程式}

\section{定数係数の非同次二階線形微分方程式の代入法}

非同次方程式の解法の一つに代入法がある。

以下の(\ref{あ})式が定数係数の非同次二階線形微分方程式である($p,q$は定数)。

\begin{equation}
\label{あ}
\displaystyle \f{\diff^2y}{\diff x^2}+p\f{\diff y}{\diff x}+qy=r(x)
\end{equation}

非同次項$r(x)$の形によって代入する関数の形が変わるため以下にまとめる。

\subsection{$\dis r(x)=Ax^n$(Aは定数)の場合}

$\dis Y(x)=a_0x^n+a_1x^{n-1}+・・・+a_{n-1}x+a_n$として(\ref{あ})式に代入して係数$a_j(j=0,1,...,n)$を決める。

\subsection{$\dis r(x)=A\cos x+B\sin x$の場合}

$\dis Y(x)=a\cos x+b\sin x$として(\ref{あ})式に代入して係数$a,b$を決める。

\subsection{$\dis r(x)=A\e^{kx}(ただし同次方程式がk^2+pk+q\neq 0)の場合$}

$\dis Y(x)=a\e^{kx}として係数を求める。$

\subsection{$\dis  r(x)=A\e^{kx}(ただし同次方程式がk^2+pk+q= 0)の場合$}

$\dis Y(x)=ax\e^{kx}として係数を求める。$同次方程式が0ということは$\dis Y(x)=A\e^{kx}$が特殊解に含まれているということなので、$\dis Y(x)にx$を掛ける。

このように代入法で0が出てきてしまったら$x$を掛ければよい。

\newpage

\section{同次二階線形微分方程式}

(\ref{あ})式で$r(x)=0$としたものが、定数係数の同次二階線形微分方程式である。$\dis y=\e^{\lambda x}$として(\ref{1})に代入すると、以下の式(同次方程式)が得られる。

\begin{equation}
\label{い}
\displaystyle \lambda^2+p\lambda+q=0 
\end{equation}

(\ref{い})式をとくと$\lambda_1$と$\lambda_2$が得られる。

\subsection{$\lambda_1 \neq \lambda_2$の場合}

一般解は$\dis y=C_1\e^{\lambda_1x}+C_2\e^{\lambda_2x}$となる。

\subsection{$\lambda_1=\lambda_2=\lambda$の場合}

一般解は$\dis y=(C_1+C_2x)\e^{\lambda x}$となる。

\section{完全微分方程式}

\subsection{全微分}

\begin{equation}
\label{all}
\displaystyle \diff u=\f{\diff u}{\diff x}\diff x+\f{\diff u}{\diff y}\diff y
\end{equation}

が全微分である。ただし、$\dis u=u(x,y)$。

\subsection{完全微分方程式}

\newtcolorbox{mysimplebox}[1]{%
 colframe=black, colback=white,
 coltitle=black, colbacktitle=white,
 boxrule=0.8pt, arc=0mm
 fonttitle=\sffamily\bfseries,
 enhanced,
 attach boxed title to top left={xshift=10mm,yshift=-3mm},
 boxed title style={frame hidden},
 title=#1}
 
\begin{mysimplebox}{完全微分方程式であるための必要十分条件}

微分方程式

\begin{equation}
\label{え}
\displaystyle \dis P(x,y)\diff x+Q(x,y)\diff y=0
\end{equation}

が完全微分方程式であるための必要十分条件は、

\begin{equation}
\label{お}
\displaystyle \dis \f{\diff P}{\diff y}=\f{\diff Q}{\diff x}
\end{equation}

である。(\ref{え})式を満たさないときは(\ref{お})式に積分因子$\mu (x,y)$をかけて、(\ref{お})式を満たさせる方法もある。

\end{mysimplebox}

全微分の問題が出た場合は上の性質((\ref{all})式または(\ref{お})式を用いて解く。

\newpage

\begin{mysimplebox}{例題}
次の微分方程式
\begin{equation}
\label{ex1}
\displaystyle -y\sin x \diff x+\cos x \diff y=0 
\end{equation}

の一般解を求めよ。(岩手大学 2021年2期)
\end{mysimplebox}

\paragraph{全微分を用いた解法}

  

(\ref{ex1})式を全微分を用いて表すと、

\begin{center}
$\dis \diff(y\cos x)$=0
\end{center}
上式の両辺を積分して、

\begin{center}
$\dis y\cos x=C (Cは任意定数)$
\end{center}
が一般解である。$\square$

\paragraph{完全微分方程式の解法}

  

(\ref{ex1})式は完全微分方程式である。したがって、求める一般解を$\dis F(x,y)$とすると、完全微分方程式は以下のように表される。

\begin{center}
$\displaystyle \diff F(x,y)=\diff F(x,y)\diff x+\diff F(x,y)\diff y=0$
\end{center}

(\ref{ex1})式から

\begin{center}
$\dis \f{\diff F}{\diff x}=-y\sin x \f{\diff F}{\diff y}=\cos x$
\end{center}

となることが分かる。両辺を積分すると、

\begin{center}
$\dis F(x,y)=y\cos x+C(y)=y\cos x+C(y)$
\end{center}

となることが分かる($\dis C(x),C(y)はそれぞれxとyの任意関数$)。

しかし、等号が成り立つには、$\dis C(x)=C(y)=C$となる必要があり、求める一般解は、

\begin{center}
$\dis y\cos x=C (Cは任意定数)$
\end{center}
である。$\square$

\newpage

\section*{あとがき}

岩手大学数物コースの院試に出る科目のまとめを作りました。間違えがあったらすみません。

\begin{thebibliography}{9}
\item 矢嶋信男著 「常微分方程式」 岩波書店

2019年時点での微分方程式の講義の指定教科書。院試の問題に章末問題からの引用が見受けられる。
  
\item 和達三樹著 「物理のための数学」 岩波書店

2020年時点での物理数学演習1の講義の指定教科書。微分方程式の代入法が載っていないが、全微分を用いた解法が載っている。

\item 上野健爾著 「応用数学」 森北出版株式会社

2020年時点での複素解析の講義の指定教科書。上の二冊に比べて新しく、分かりやすい。
  
\end{thebibliography}





















\end{document}